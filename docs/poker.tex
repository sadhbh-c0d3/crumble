\documentclass[11pt, a4paper]{article}

\usepackage[utf8]{inputenc}
\usepackage{amsmath, amssymb, amsthm}
\usepackage{hyperref}
\usepackage{geometry}
\usepackage{graphicx}
\usepackage{algorithmic}
\usepackage{enumitem}

\geometry{margin=1in}

\title{\textbf{The Sovereign Referee Protocol (SRP): \\ A Stateless Architecture for Trustless Peer-to-Peer Poker}}
\author{Sonia Code \& Gemini}
\date{February 2026}

\begin{document}

\maketitle

\begin{abstract}
For 47 years, the realization of a purely decentralized, dealer-less card game has been constrained by the computational overhead of verifiable multi-party computation. This paper introduces the Sovereign Referee Protocol (SRP), a stateless, peer-to-peer architecture designed for Arbitrum Stylus. By replacing traditional Merkle Trees with dynamically scaled Lagrange Polynomial Commitments over the BLS12-381 curve, SRP compresses the entire game state into a single 48-byte $G_1$ root. This document details the exact state transitions of a Sovereign Poker game, demonstrating how commutative encryption, KZG proofs, and an economic heartbeat combine to eliminate the trusted house.
\end{abstract}

\section{Introduction}
The theoretical foundation for dealer-less card games was established in 1979 with the Mental Poker algorithm. However, moving complete 52-card state histories on-chain has historically destroyed commercial viability due to gas constraints. 

SRP resolves this by treating the blockchain strictly as a "Stateless Referee." Utilizing the custom \texttt{crum\_bls} library, the protocol offloads all state management to the peer-to-peer layer. Furthermore, SRP introduces Dynamic Degree Scaling, an optimization that completely truncates "dead" cards, ensuring cryptographic energy is spent exclusively on the cards in active play.

\section{Core Architecture}

\subsection{Pillar 1: The Commutative Mask}
All cards are represented as points on the $G_1$ elliptic curve. Each player generates a secret scalar $sk_i$. Because scalar multiplication on the curve is commutative, players can apply and remove their cryptographic locks in any order without corrupting the underlying card point.

\subsection{Pillar 2: Dynamic Degree Scaling \& Polynomial Anchors}
A standard Texas Hold'em hand requires exactly $K = 5 + 2N$ cards, where $N$ is the number of players. Anchoring a full 52-card Merkle Tree wastes computation. Instead, SRP treats the active deck as a mathematical curve. 

Players interpolate a Lagrange polynomial $D(x)$ of degree $K-1$, where evaluating the polynomial at index $i$ yields the fully masked card $C_i$. Using a KZG commitment, this entire polynomial is compressed into a single $G_1$ point ($Root$), which is submitted to Arbitrum.

\subsection{Pillar 3: The Stateless Audit}
Verifying a card peel requires zero sibling-hashes. A player simply provides their unmasked point and a single $G_1$ evaluation proof ($\pi$). The Arbitrum Stylus contract validates the polynomial proof using an $O(1)$ Miller Loop (Bilinear Pairing), comparing it against the anchored $Root$ and the player's $G_2$ Public Key.

\subsection{Pillar 4: The Economic Heartbeat}
SRP enforces liveness through economic slashing. Every cryptographic transition is bound by a 120-second timeout. If a player fails to provide their unmasking scalar, their stake is forfeited to the honest participants via the smart contract.

\section{The Sovereign Game Lifecycle}

\subsection{1. Player Joins the Game (Staking \& Registration)}
\begin{itemize}
    \item A player calls \texttt{join\_table()} on the Arbitrum Stylus contract, depositing their USDC stake.
    \item The player registers their $G_2$ Public Key ($PK_i$), establishing their verifiable identity for the Miller Loop audit.
\end{itemize}

\subsection{2. The Truncated Sovereign Shuffle}
The deck begins as a sorted vector of 52 $G_1$ points.
\begin{itemize}
    \item \textbf{Shuffle-and-Mask:} Player 1 masks all 52 points with $sk_1$, randomly permutes the vector, and passes it to the next player. This repeats until Player $N$ finishes.
    \item \textbf{Truncation:} The fully locked, fully shuffled 52-card vector is truncated to the top $K$ points required for the specific game format (e.g., $K=9$ for Heads-Up Hold'em).
    \item \textbf{Commitment:} Players compute the Lagrange polynomial $D(x)$ for these $K$ points. The resulting KZG $Root$ ($G_1$ point) is submitted to the Stylus contract as the immutable anchor for the hand.
\end{itemize}

\subsection{3. Posting Blinds \& Pre-Flop Deal}
\begin{itemize}
    \item Players agree on blind deductions via P2P threshold signatures (\texttt{crum\_bls::lagrange::combine}), committing the new balance state.
    \item To view hole cards (indices 0 and 1), Player 1 requests the unmasking values from all other players. The other players transmit $U_0$ and $U_1$ by applying their modular inverses ($sk_i^{-1}$). Player 1 applies their final inverse to secretly reveal the cards.
\end{itemize}

\subsection{4. Betting Rounds \& Community Cards (Flop, Turn, River)}
\begin{itemize}
    \item \textbf{Betting:} Players broadcast actions (Call, Raise, Fold) P2P. Active players sign the updated pot state, creating a cryptographically secure Hand History.
    \item \textbf{Dealing:} For the Flop (indices 2, 3, 4), all active players broadcast their inverse scalars simultaneously. The three $G_1$ points are publicly reconstructed. This process repeats for the Turn and River.
\end{itemize}

\subsection{5. The Showdown}
Remaining players reveal their private hole cards by broadcasting their final unmasking scalars. The table verifies that the revealed cards mathematically correspond to the KZG polynomial $Root$. The winner submits the threshold-signed final state to the Stylus contract for instant USDC settlement.

\section{The Unhappy Path: Malice and Disconnections}
If a losing player attempts to stall the game by withholding an unmasking scalar:
\begin{enumerate}
    \item \textbf{Dispute Initiation:} The honest player submits the threshold-signed game state to Arbitrum Stylus.
    \item \textbf{The Timer:} A 120-second countdown begins, demanding the stalling player submit their unmasking point directly on-chain.
    \item \textbf{The Audit:} If submitted, the contract runs the \texttt{crum\_bls} Miller Loop against the polynomial $Root$.
    \item \textbf{Slashing:} If the pairing fails (the math is manipulated) or the timer expires, the malicious player is slashed. Their USDC is distributed to the honest players, and the hand terminates.
\end{enumerate}

\section{Conclusion}
By converging 1979 Mental Poker theory with 2026 KZG polynomial commitments and Arbitrum Stylus, the Sovereign Referee Protocol achieves $O(1)$ verification costs and absolute autonomy. It provides the definitive architecture for a mathematically enforced, dealer-less financial ecosystem.

\end{document}